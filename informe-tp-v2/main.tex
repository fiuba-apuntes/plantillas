\documentclass[a4paper,12pt,twoside]{article}
\usepackage[spanish]{babel}
\usepackage[utf8]{inputenc}
\usepackage{graphicx} % para insertar graficos/imagenes
\usepackage{float} % me deja usar la H de 'here' en los graficos para ponerlos donde yo quiera
\usepackage{fancyhdr} % headers y footers
\usepackage{color} %Para colores y eso
%\usepackage{geometry} %Para cambiar la geometria de las hojas

\newcommand{\facultad}{Facultad de Ingeniería}
\newcommand{\tituloTP}{Uso del Gradiente Extendido}
\newcommand{\nroTP}{5}
\newcommand{\codigoMateria}{79.30}
\newcommand{\nombreMateria}{Matemática para Ingenieros Especiales}
\newcommand{\departamento}{Departamento de Gestión}

% Cosas para la caratula
\newcommand{\txtFacultad}{\Large{\textsc{\facultad}}}
\newcommand{\txtNombreTP}{TP \nroTP}
\newcommand{\txtTituloTP}{\tituloTP}
\newcommand{\txtFecha}{\today}
\newcommand{\txtMateria}{\codigoMateria -- \nombreMateria \newline \departamento}

\newcommand{\txtRotulo}{
	\begin{minipage}[t][10cm]{\textwidth}
		\centering
		\large{\textsc{\txtNombreTP}}
		
		\huge{\textbf{\tituloTP}}
		
		\hspace*{1cm}
		
		\normalsize
		\txtFecha
		
		\txtMateria
	\end{minipage}
	}

\newcommand{\txtCorrecciones}{\quad}
\newcommand{\txtAutores}{
	\begin{minipage}[b][10cm][t]{\textwidth}
	\vfill
	\begin{tabbing}
		\underline{Autores:}	\hspace{-1cm}\=\+\hspace{3cm}\=\hspace{3cm}\=\\[.5cm]
		Apellido, Nombre  \\
		\> Padrón 00000 -\> \textit{E-mail:} lalala@gmail.com\\[.3cm]
		Apellido, Nombre  \\
		\> Padrón 00000 -\> \textit{E-mail:} lalala@gmail.com\\[.3cm]
		Apellido, Nombre  \\
		\> Padrón 00000 -\> \textit{E-mail:} lalala@gmail.com\\
	\end{tabbing}
\end{minipage}
}

% Defino el header y footer predeterminado
\pagestyle{fancy} % seleccionamos un estilo
\fancyhead{}
\fancyfoot{}
\lhead{\facultad} % texto izquierda de la cabecera
\rhead{\nombreMateria \, (\codigoMateria)} % texto centro de la cabecera
\cfoot{\thepage}
% %
\usepackage{anysize} % me permite definir los margenes como quiera
\marginsize{2cm}{2cm}{1cm}{1.5cm} %izquierda, derecha, arriba, abajo



% Cosas para poner la caratula desde Inkscape
\usepackage{eso-pic} % Permite poner una imagen de fondo

% Sarta de cosas para que compile el SVG a PDF+TeX
% (es necesario que Inkscape esté en PATH)
\newcommand{\executeiffilenewer}[3]{%
	\ifnum\pdfstrcmp{\pdffilemoddate{#1}}%
	{\pdffilemoddate{#2}}>0%
	{\immediate\write18{#3}}\fi%
}
\newcommand{\includesvg}[1]{%
	\executeiffilenewer{#1.svg}{#1.pdf}%
	{inkscape -z -D --file=#1.svg %
		--export-pdf=#1.pdf --export-latex}%
	\input{#1.pdf_tex}%
}
% %


\begin{document}
	\begin{titlepage}
		\thispagestyle{empty} % Elimino header y footer
%		\newgeometry{top=0cm,left=0cm,bottom=0cm,right=0cm}
		
\AddToShipoutPicture*{ %Pongo la caratula en 0,0
			\put(0,0){%
				\parbox[b][\paperheight]{\paperwidth}{%
					\vfill
					\centering
%					\def\svgwidth{\pagewidth}
					\includesvg{caratula_tps}
					\vfill
				}}
}
		\quad
		\vfill
		
	\end{titlepage}
		\restoregeometry
		\ClearShipoutPicture

	
	\section{Licencia de Uso de esta Plantilla}
Este modelo está pensado para usar como plantilla para los Trabajos Prácticos de los estudiantes de la Facultad de Ingeniería de la Universidad de Buenos Aires.

De aquí en adelante, por ``esta obra'' se deberá entender el código fuente y los diseños correspondiente, a excepción de los logotipos de la Facultad de Ingeniería de la Universidad de Buenos Aires y de la Universidad de Buenos Aires. Asumimos que tiene derecho a usarlos, si no es así usted puede editar el PDF (con, por ejemplo, Inkscape) para borrarlos o cambiarlos por los que sean correspondientes y tenga derecho a usar.

Bajo lo mencionado anteriormente, usted puede utilizar esta obra para sus trabajos prácticos, informes o lo que sea que necesite, bajo su propio riesgo y sin garantía. No tiene necesidad de dar ninguna atribución al respecto y es libre de hacer con ella lo que bien le parezca. 

Si tiene alguna mejora, puede hacer un \textit{pull request} al repositorio de GitHub del Proyecto Fiuba Apuntes o comunicarse con nosotros. Encontrará más información en el sitio web: http://fiuba-apuntes.github.io/

Al usar esta obra, usted está aceptando estos Términos y Condiciones de Uso.
	
\end{document}