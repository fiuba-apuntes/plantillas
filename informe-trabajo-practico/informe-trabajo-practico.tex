\documentclass[a4paper]{article}
\usepackage[utf8]{inputenc} % para poder usar tildes en archivos UTF-8
\usepackage[spanish]{babel} % para que comandos como \today den el resultado en castellano
\usepackage{a4wide} % márgenes un poco más anchos que lo usual
\usepackage[showRevisiones]{caratula}

\begin{document}

\materia{Materia del Apunte}
\tipoapunte{Tipo de Apunte (Teórico o Práctico)}

\fecha{\today}

\tema{Tema de la Materia}
\subtema{Subtema}

\autor{Apellido, Nombre1}{001/01}{email1@dominio.com}
\autor{Apellido, Nombre2}{002/01}{email2@dominio.com}
% Pongan cuantos integrantes quieran

\revision{04/01/2015}{Apellido, Nombre}{Primera versión del apunte}
\revision{05/01/2015}{Apellido, Nombre}{Segunda versión del apunte, se cambió tal y tal cosa.}
\maketitle

Contenido del apunte...

\end{document}
